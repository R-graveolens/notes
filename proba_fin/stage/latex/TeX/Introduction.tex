% TeX root = ../Main.tex
\section{Introduction}
Although the Black-Scholes-Merton (BSM) model's deviation from reality in its assumptions is widely recognized, 
practitioners continue to prefer the BSM formula. This is primarily due to its 
function as a practical mapping tool, transforming option prices into a singular 
real scalar known as implied volatility (IV), which reveal significant market 
information.

When plotting implied volatilities against strike prices at a consistent 
maturity, a common observation is a skew or smile pattern. 
This pattern has been demonstrated to correlate directly with the conditional 
non-normality of the underlying return risk-neutral distribution. 
Specifically, a smile indicates fat tails in the return distribution, 
while a skew signifies asymmetry in the return distribution. 
Moreover, variations in the implied volatility smile across option maturity 
and calendar time provide insights into how the conditional return distribution 
non-normality changes across different conditioning horizons and over different time periods.

Constructing a well-defined, arbitrage-free implied volatility surface from market price data has always been an important and pressing issue.
The concept of arbitrage-free can be approached from two perspectives. 
One is to examine the characteristics of the implied volatility surface obtained directly from price data and the conditions it imposes on implied volatility for arbitrage-free. 
The other perspective involves how to impose arbitrage-free conditions in some way on the price surface to obtain a smooth and arbitrage-free surface. 
It's important to note that we're primarily considering static arbitrage here.
