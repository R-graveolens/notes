\chapter{Setting and notation}
In this study, we operate within the multicurve interest rate market, focusing primarily on two key interest rate indices. 
The Euro Interbank Offered Rate (Euribor, EIB) reflects the 1-month, 3-month, 6-month, and 12-month interest rates at which 
banks lend unsecured funds to each other within the eurozone. Additionally, the eurozone's risk-free rate, the 
Euro Short-Term Rate (ESTER, EST), will also be considered.

In terms of the instruments used in this study, the Zero-Coupon bond (ZC) is an instrument that pays 1 unit of currency 
at maturity. The ZC curve, also known as the discount curve, is currently dependent solely on the risk-free rate, 
i.e., the EST. We denote \( B(t,T) \) as the present value at time \( t \) of a ZC bond that pays 1 unit of currency 
at time \( T \).
An interest rate swap is a contract between two parties in which they agree to exchange a predetermined fixed interest rate for a floating 
interest rate (potentially based on different rate indices) at regular intervals over an agreed period, based on a notional amount. 
The most common swap product in the market is the plain vanilla swap, typically structured as an exchange of fixed interest rate payments for 
floating rate payments.
For the fixed leg of the swap, all cash flows are determined at the time the swap is initiated. The cash flow at time \(T_i\) is calculated 
by multiplying the accrual factor \(\delta_{i}^{f}\), which corresponds to the period \([T_{i-1}^{f}, T_{i}^{f}]\), by the agreed fixed rate.
For the floating leg, cash flows are determined throughout the life of the swap. On each payment date, starting at \(T_{1}^{v}\), the cash flow 
at \(T_i^{v}\) is calculated at \(T_{i-1}^{v}\), where the accrual factor \(\delta_{i}^{v}\) is multiplied by the floating index rate 
\(I(T_{i-1}^{v}, T_{i-1}^{v}, T_{i}^{v})\), which represents the floating rate for the period \([T_{i-1}^{v}, T_{i}^{v}]\) as observed at 
time \(T_{i-1}^{v}\).
A forward swap, on the other hand, is an agreement made at time \(t\) to enter into a swap starting at \(T_0\) and ending at \(T_n\) in the 
future. The forward swap rate is the fixed rate agreed upon between the counterparties when entering the forward swap, 
such that the present value of the fixed and floating legs of the forward swap are equal. In other words, it is the fixed swap rate that makes 
the present value of the swap equal to zero. For a forward swap starting at a future date \(T_0\), the forward swap rate agreed upon at present 
(time \(t\)) is denoted as:
$$
R(t, T_{0}, T_{n}) = ...
$$
If the floating rate \(I\) and the Zero-Coupon bond \(B\) are based on the same index, i.e., if \(I\) is also based on the EST, 
we can simplify and express the forward swap rate as:
$$
R(t, T_{0}, T_{n}) = ...
$$

A swaption is an option where the underlying security is a forward swap. In line with swap terminology, 
a call swaption is also referred to as a payer swaption, while a put swaption is called a receiver swaption. 
Even though multiple yield curves or indices may exist for the same currency, they share the same volatility surface. 
This volatility surface is calibrated using market prices of swaptions. Using the notion of the swap rate, we can express the present value of 
a payer swaption with a fixed strike rate \(K\) as:
$$
\mathbf{PV}(t) = LVL(t, T_{0}, T_{n})(R(t, T_{0}, T_{n}) - K)_{+}
$$
where
$$
LVL(t, T_{0}, T_{n}) = ...
$$
represents the level or PVBP (Price Value of a Basis Point) of the swap. 
According to martingale pricing theory, using \(LVL\) as the numeraire, and after performing the appropriate probability change, 
the price of a payer swaption with an expiry date of \(T\) and strike \(K\) is given by:
$$
C_{t}(K,T) = LVL(t, T_0, T_n)\mathbb{E}^{Q^{LVL}}_{t}[(R(T, T_0, T_n)-K)_{+}],
$$
where \(LVL(t, T_0, T_n)\) represents the level or Price Value of a Basis Point (PVBP) for the underlying swap, and \(\mathbb{E}^{Q^{LVL}}_{t}\) 
denotes the expectation under the forward measure \(Q^{LVL}\), which corresponds to the numeraire \(LVL\).
Additionally, under this probability measure \(Q^{LVL}\), the swap rate \(R(T, T_0, T_n)\) behaves as a martingale. 

A Constant Maturity Swap (CMS) is a type of interest rate swap where the floating leg is linked to a swap rate with a fixed maturity, 
rather than a standard money market rate like Euribor. Building on this concept, a CMS forward is a forward contract based on the future 
value of a CMS rate. Similar to a traditional forward rate agreement (FRA), a CMS forward locks in the swap rate for a future period. 
However, instead of focusing on short-term interest rates, a CMS forward specifically targets the constant maturity swap rate, 
enabling market participants to hedge or speculate on the movement of long-term swap rates over a defined period.
The price of a CMS forward encapsulates all relevant information about the payer and receiver swaptions in the market, 
and this can be understood from two perspectives. First, the pricing formula for CMS instruments can be expressed as an integral over 
the distribution of vanilla swaptions. Second, the CMS forward rate reflects the market's expectations of future swap rates, 
which are influenced by the pricing of both payer and receiver swaptions. These swaptions capture the market's view on interest rate 
volatility and directional movements, making the CMS forward price a comprehensive measure that incorporates implied volatility and 
risk-neutral probabilities derived from swaption markets. 
From a probabilistic perspective, the price of the CMS forward can be written as:
$$
\mathbb{E}^{Q_{T}}[R(T, T_0, T_n)],
$$
where \( Q_T \) is the forward measure associated with the maturity \( T \), and \( R(T, T_0, T_n) \) is the forward swap rate at time \( T \). 
Under the swap measure \( Q^{LVL} \), this expression can be transformed as:
$$
\mathbb{E}^{Q_{T}}[R(T, T_0, T_n)] = \frac{LVL(0)}{B(0, T)} \mathbb{E}^{Q^{LVL}} \left[\frac{R(T, T_0, T_n)}{LVL(T)}\right].
$$





