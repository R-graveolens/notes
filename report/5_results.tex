\chapter{Experiments and Results}
Regarding the data source, we did not use actual market prices for this analysis. Instead, we generated swaption prices using the Ho-Lee model and introduced volatility shocks at six points of moneyness \((-2, -1, 0, 1, 2, 10)\). This process allowed us to reprice the swaptions and derive the bid, mid, and ask quotes for our study.

As shown in Figure \ref{...}, the algorithm effectively captures even highly complex smile structures. Throughout this chapter, the graphs display several colorful curves: the purple curve represents the original market swaption prices, while the yellow, blue, and green curves correspond to the ask, mid, and bid quotes after applying the market shocks. The red curve represents the results produced by our model, demonstrating its ability to align with and adapt to various market conditions, even in the presence of significant volatility shifts.

There is also an interesting observation worth mentioning here. As noted earlier, we can solve the original optimization problem using the Newton method, and we conducted tests with various tolerance parameters. The results indicate that with a smaller tolerance, the Newton method produces outcomes similar to those obtained from the Sinkhorn algorithm, but with a longer execution time. Conversely, when a larger tolerance is set, the execution time decreases, albeit with a slight deterioration in accuracy. One of the result comparisons, along with the execution time analysis, is included in the Appendix.

Additionally, after conducting several experiments with both weight choices, we observed that even with spread weights, the algorithm can achieve good results, maintaining control across the entire range of strikes. However, in some cases, control in the deep out-of-the-money region is not always necessary, and focusing on all strike ranges can sometimes introduce numerical instability. Therefore, in the following experiments, we primarily use vega weights, although spread weights remain effective in these scenarios as well.


This chapter is divided into two main sections. In the first section, 
we present the results of the original arbitrage-free calibration. 
Subsequently, the second section will showcase the results of the joint calibration problem.

\section{Original Calibration Experiments}
In these experiments, we focus on swaptions with a 1-month expiry and a 10-year tenor.
In the algorithm's settings, we need to select the weights \(\omega_{K}\) as a penalty for each price at each strike. Here, we define:
\[
\omega_{K} = \alpha \times \Delta_{K},
\]
where \(\alpha\) is a hyperparameter that is set at the beginning of the algorithm. For \(\Delta_{K}\), there are two ways to define it:
\begin{itemize}
    \item Vega Weights: \(\Delta_{K} \propto \text{Vega}_{K}\) in the Bachelier model. In this case, a reasonable range for \(\alpha\) is \([1e-5, 1e-9]\).
    \item Spread Weights: \(\Delta_{K} = c^{\text{ask}}_{K} - c^{\text{bid}}_{K}\), where \(\alpha\) typically falls within the range \([0.1, 1e-5]\).
\end{itemize}
In every experiment conducted in this section, satisfactory results were typically achieved after about 50 iterations, 
and in most cases, the process took no more than 3 minutes, often just under one minute, 
when using Python code on a standard personal computer.

Firstly, as shown in Figure \ref{hat}, this algorithm demonstrates the capability to capture even highly complex smile structures.

In the following analysis, we fixed the scenario and gradually decreased \(\alpha\), which controls the weights \(\omega\) 
for both types of weights. Figure \ref{alpha_spread} shows the results using spread weights, while Figure \ref{alpha_vega} 
presents the outcomes for vega weights.

From these two graphs, we observe that as the weight \(\omega\) becomes smaller (i.e., as \(\alpha\) decreases), 
the implied volatility calculated by the model becomes closer to the implied volatility of the mid prices. 
Conversely, with larger weights, the smile structure appears smoother, and the resulting RND is more regular.

For the results using vega weights, we notice that even with a small \(\alpha\), the swaptions with out-of-the-money 
strikes are not well-calibrated. This is because, at these strikes, the vega values are large, which effectively 
reduces the impact of the constraint imposed by \(\alpha\).

Now, we fix \(\alpha\) to be as small as possible in each case to closely match the volatility of the mid prices. 
Meanwhile, we gradually increase the impact on the market to introduce more potential arbitrage. 
Despite this, we observe that the RND produced by the model remains free of arbitrage, 
and its shape closely resembles that of the market.

Even though the vega weights might lose some control in the out-of-the-money (OTM) region, 
we can conclude that the algorithm performs effectively in eliminating arbitrage while maintaining a close 
alignment with the market's implied volatility curve in both weighting approaches.

The final result presented in this section is the test involving bid-ask spreads, where we gradually increase the bid-ask quotes 
to evaluate the performance of the algorithm under these conditions. 
In this experiment, we fixed \(\alpha\) at a relatively large value; otherwise, the algorithm would consistently align with the 
mid-quote curve.

From the test results, we observe that not only does \(\alpha\) control the degree of calibration, but the bid-ask quotes themselves 
also have an impact on the outcome. Even in cases where the bid-ask spreads are significantly wide, 
the calibrated implied volatility curve remains confined within this bid-ask range, demonstrating the algorithm's robustness in 
maintaining the bid-ask quotes condition.

\section{Joint calibration experiments}
In this set of experiments, we studied swaptions with 4-year expiry and 10-year tenor, using only vega weights for calibration. 
We present three different scenarios, each illustrated with two graphs: the first graph shows the volatility and RND plot, 
while the second graph displays the price curve along with a visualization of the differences between the prices generated by the model 
and the market prices (for both swaptions and the CMS forward).

The results of the first scenario, shown in Figure \ref{CMS1_vol} and Figure \ref{CMS1_diff}, 
highlight the trade-off between the two approaches when handling the joint calibration problem. 
The first method produces an arbitrage-free posterior RND capable of accurately reproducing the CMS forward price 
(with a difference of approximately \( \sim 1e-17 \)). However, it struggles to keep the calibrated swaption prices close to the market values, 
often failing to stay within the bid-ask range. 

In contrast, the second method manages to maintain the swaption prices in line with the market values, 
but the difference in the CMS forward price is noticeably larger (although still within 1 basis point). 
In other scenarios, the first method occasionally remains close to the market prices and within the bid-ask region, 
but the discrepancies between swaption prices are consistently larger than those observed with the second method.
















\begin{comment}
    \chapter{Results and variations}
This section provides an overview of the numerical outcomes generated by the methods we have introduced. We commence with a straightforward illustration from the example outlined in paper \cite{mission}, where in the focus lies on elucidating the process of designing neural network structures and selecting hyperparameters. In the subsequent example, we extend this algorithm to a specific interest rate product under more intricate conditions, subsequently comparing the outcomes with benchmark methods. Lastly, we propose a more generalized model and show its results.

\section{Application on 1D vanilla American put option}
Let's consider American options within the framework of the Black-Scholes Model. The price of this asset adheres to the dynamic described by $X_{t}=S_{t}$: $$dS_{t}=rS_{t}dt+\sigma S_{t}dW_{t}.$$ and the discounted factor is $\beta_{t}=\exp(rt)$. Consequently, the core formula for this scenario can be simplified to 
$$ \frac{V_{t_{i+1}}}{\beta_{\Delta t}} \approx 
        \Phi_{t_{i}}(S_{t_{i}})+
        \beta_{t_{i}}(\Psi_{t_{i}}(S_{t_{i}})\Delta W_{t_{i}} + \sigma \Psi^{'}_{t_{i}}(S_{t_{i}})(\frac{\Delta W_{t_{i}}^{2}-\Delta t}{2}))$$

In the first example, we assess the pricing of a one-dimensional (1D) vanilla American put option. This option matures at $T = 1$ and holds a strike price of $K = 40$, with an initial price of $S_{0} = 36$. The associated parameters include a risk-free rate, $r = 0.06$, and a volatility factor, $\sigma = 0.2$. The payoff function at time $t$ is represented as $Z(S_{t},K) = (K-S_{t})^{+}$.

A crucial point lies in the method of constructing the neural network model within the black-box framework. Through exploratory experiments, we have identified an effective strategy involving the utilization of two mutually independent networks to separately approximate these two functions, as depicted in the following graph \ref{NN}. This approach proves advantageous, as martingale increment and continuation value functions can exhibit diverse complexities, particularly in intricate models characterized by higher dimensions.
\begin{figure}[ht]
\includegraphics[scale=0.4]{images/NN.png} 
\caption{The Structure of Neural Networks}
\label{NN} 
\end{figure}

When it comes to selecting appropriate hyperparameters during the training process, we rely on the Python library \textbf{Optuna} to obtain values that are relatively optimal. The validation loss values derived from various hyperparameter combinations are presented in Figure \ref{optuna1}, while the importance ranking of these hyperparameters is shown in Figure \ref{optuna2}.

\begin{figure}[!h]
    \centering
    \includegraphics[scale=0.2]{images/optuna1.png}
    \caption{The outcomes from combinations of hyperparameters}
    \label{optuna1}
\end{figure}

\begin{figure}[!h]
    \centering
    \includegraphics[scale=0.3]{images/optuna2.png}
    \caption{The importance indices for the hyperparameter}
    \label{optuna2}
\end{figure}

To conduct the results, we discretize the time interval $[0, t]$ into 50 sub-intervals and employ $10^{5}$ paths to compute the price of this option. With the neural networks has structure  $[(20,20),(20,20,20)]$, and learning rate is set at $0.0008$, the batch size is designated as $1000$, and the training proceeds for a total of $200$ epochs. 
When executing Algorithm 1 and training the model in each step, the previously trained network parameters can be utilized as initial values for the next model. This can save a lot of training time. This is justifiable due to the analogous shapes exhibited by both the continuation functions and martingale increment functions at different time instances. Given the resemblances observed in \cite{mission}, the parameters of corresponding networks are expected to bear similarities as well.

Ultimately, we have obtained results that closely align with those presented in the paper $(4.4762 \  \text{v.s.} \ 4.4887)$. Additionally, it is evident that the Milstein Scheme outperforms the Euler Scheme, underscoring the superiority of employing the second-order approximation inherent to the Milstein Scheme. Consequently, we opt to standardize our algorithm utilizing the Milstein Scheme for enhanced performance which is different from the paper who use the Euler Scheme in the Algorithm.
\begin{table}
    \centering
    \begin{tabular}{lccc}
        \hline
         &Lower Bound  &Upper Bound  &Difference\\
        \hline
        Euler Scheme &4.467572  &4.576317 &0.108745\\
        \hline
        Milstein Scheme &4.475878  &4.491022  &0.015144‬\\
        \hline
    \end{tabular}
\caption{Results on Equity}
\label{Res_equity}
\end{table}

\section{Application on Interest Rate Products}
In this particular example, we extend the application of our algorithm to address the pricing of interest rate options under the Hull-White model. For the sake of comparative analysis, we select the PDE Pricer and the Rogers' method as benchmarks to evalue the results obtained. Lastly, we conclude by introducing a broader perspective into the model by introducing randomization to both the spot rate $r$ and $\sigma$, thus achieving a more comprehensive framework.

\subsection{Hull White model and Bermudan Floor}
The employed market-simulation model is the Hull-White model, a specification within the HJM framework that provides a comprehensive representation of the forward rate curve. Thus, we operate within a filtered probability space $(\Omega, \mathcal{F}, \mathbb{Q}, (\mathcal{F}_{t})_{t \in [0,T_{n}]})$, where $T_{n} > 0$ designates a fixed finite horizon. Under an equivalent martingale measure $\mathbb{Q}$, the dynamics $P_{t,T}$ of the price at time $t$ for a zero-coupon bond maturing at $T$ are expressed as:
\begin{equation}\label{zc}
dP_{t,T} =P_{t,T}(r_{t}dt+\sigma(t)\frac{1-e^{-\lambda(T-t)}}{\lambda}dW_{t}^{\mathbb{Q}})    
\end{equation}

Here, $r_{t}$ denotes the instantaneous risk-free short rate, $\lambda$ represents the mean-reversion parameter, and $W_{t}^{\mathbb{Q}}$ signifies a Brownian motion under $\mathbb{Q}$. We introduce the notation:
$$
\gamma_{t,T,\lambda}=\sigma(t)\eta_{t,T,\lambda} = \sigma(t)\frac{1-e^{-\lambda(T-t)}}{\lambda}
$$

The solution to equation \ref{zc} is obtained as:
$$
P_{t,T} = P_{0,T} \exp(\int_{0}^{t}(r_{s}-\frac{1}{2}\gamma_{s,T,\lambda}^{2})ds + \int_{0}^{t}\gamma_{s,T,\lambda}dW_{s}^{\mathbb{Q}})
$$
Note that, throughout this report, we assume that $\sigma(t) = \sigma$ is constant.

By formulating the solution for the process $P_{s,t}$, expressing the ratio $\frac{P_{t,T}}{P_{s,t}}$, and taking the limit as $s \rightarrow t$, after computations, we derive an expression for $P_{t,T}$ that don't depend on the process $r_{t}$:
$$
P_{t,T} = \frac{P_{0,T}}{P_{0,t}}\exp (-\frac{\sigma^{2}}{2}\eta_{t,T,\lambda}(\eta_{0,t,\lambda}^{2}+\eta_{t,T,\lambda}\eta_{0,t,2\lambda}))\exp(\eta_{t,T,\lambda}\int_{0}^{t}\sigma e^{-\lambda(t-s)}dW_{s}^{\mathbb{Q}})
$$

Now we pass the measure $\mathbb{Q}$ denoted the risk-neutral probability measure, which is associated with the numeraire $\beta_{t} = e^{\int_{0}^{t}r_{s}ds}$ to the forward measure $\mathbb{Q}^{T}$ represented the forward-neutral probability measure of maturity, linked to the zero-coupon numeraire $P_{t,T}$. So,
$$
\frac{d\mathbb{Q}^{T}}{d\mathbb{Q}}|_{\mathcal{F}_{t}} := \frac{\beta_{0}P_{t,T}}{\beta_{t}P_{0,T}}
$$
Through the application of Girsanov's theorem, the transition from the forward measures to the risk-neutral measure is represented as: 
$$dW_{s}^{\mathbb{Q}^{T}} = dW_{s}^{\mathbb{Q}}-\gamma_{s,T}ds.$$

Furthermore, the transition between forward measures is also established as:
$$dW_{s}^{\mathbb{Q}^{T}}+\gamma_{s,T}ds = dW_{s}^{\mathbb{Q}^{T^{'}}}+\gamma_{s,T^{'}}ds.$$

Shift our focus to the pricing measure $\mathbb{Q}^{T_{n}}$, where $T_{n}$ is the horizon date.

Consequently, we arrive at 
\begin{equation}
   P_{t,T} = \frac{P_{0,T}}{P_{0,t}}\exp (\sigma^{2} \eta_{t,T,\lambda} \eta_{t,T_{n},\lambda} \eta_{0,t,2\lambda}-\frac{1}{2}\sigma^{2}\eta_{t,T,\lambda}^{2}\eta_{0,t,2\lambda})\exp(\eta_{t,T,\lambda}X_{t}). 
\end{equation}
where $X_{t} = \int_{0}^{t}\sigma e^{-\lambda(t-s)}dW_{s}^{\mathbb{Q}^{T_{n}}}$ represents an abstract diffusion process under measure $\mathbb{Q}^{T_{n}}$. It is worthy to note that diffusing the process $X_{t}$ enables us to diffuse all zero-coupon bonds of various maturities at time $t$, as the remaining components of the formula are deterministic. Moreover, the process $X_{t}$ is a Wiener integral, making it a Gaussian process:
$$
X_{t} \sim \mathcal{N}(0,\int_{0}^{t}\sigma^{2}e^{-2 \lambda(t-s)}ds) = \mathcal{N}(0,\sigma^{2}\eta_{0,t,2\lambda}).
$$
For the sake of convenience, we conduct our example on a non-existent product - Bermudan Floor. In this context, $ X_{t}$ represents the diffusion process, while $P_{t,T_{n}}(X_{t})$ serves as the Numeraire, denoting the discounted factor. The payoff for the Bermuda Floor is denoted as 
$$
Z_{T}=P_{T,T+\theta}*(L(T,T+\theta)-K)^{+}*\theta,
$$
where $T$ is the maturity of the option, $\theta$ represents the time between the exercise date and the settlement date, $T_{n}=T+\theta$ is the horizon date while $K$ is the strike. And $ L(T,T+\theta) = (\frac{1}{P(T,T+\theta)}-1)\frac{1}{\theta} \ \text{is the LIBOR.}$
So the pricing problem is 
\begin{align*}
V_{0} &= \sup_{\tau \in\mathcal{T}} \mathbb{E}^{Q}[e^{-\int _{0}^{\tau+\theta}r_{s}ds}\theta(K-L(\tau,\tau+\theta))_{+}]\\
&=\sup_{\tau \in\mathcal{T}} P_{0,T_{n}}\mathbb{E}^{Q^{T_{n}}}[\frac{Z_{\tau}}{P_{\tau,T_{n}}}]
\end{align*}
where $Z_{t} &= P_{t,t+\theta}\theta(K-L(t,t+\theta))^{+}$.
Thus, the core formula in this case become:
$$
  P_{t_{i},T_{n}}\frac{V_{t_{i+1}}}{P_{t_{i+1},T_{n}}} \approx \Phi(X_{t_{i}})+P_{t_{i},T_{n}}(\Psi(X_{t_{i}})\Delta W_{t_{i}} + \sigma \Psi^{'}(X_{t_{i}})(\frac{\Delta W_{t_{i}}^{2}-\Delta t}{2})).
$$
\subsection{Benchmarks}
Before delving into the presentation of results, let us introduce two conventional pricing methods to serve as benchmarks.
\subsubsection{PDE pricer}
Let's consider a European option over the interval $[t,T]$ and denote its value function as $\pi$. We have:
\begin{align}
    \pi(x,t) &=\mathbb{E}^{\mathbb{Q}}[e^{-\int_{t}^{T}r_{u}ds}Z(X_{T})|X_{t}=x]\\
    &=P_{t,T_{n}}\mathbb{E}^{\mathbb{Q}^{T_{n}}}[\frac{Z(X_{T})}{P_{T,T_{n}}}|X_{t}=x]
\end{align}
Let $u(x,t) = \mathbb{E}^{\mathbb{Q}^{T_{n}}}[\frac{Z(X_{T})}{P_{T,T_{n}}}|X_{t}=x]$.
Since $X_{t}$ follows an Itô process, we have:
$$dX_{t} = -\lambda X_{t}dt + \sigma dW_{t}^{\mathbb{Q}^{T_{n}}}.$$
According to the Feynman-Kac theorem, $u$ satisfies the following partial differential equation:
$$\frac{\partial u}{\partial t}(x,t)-\lambda x\frac{\partial u}{\partial x}+\frac{1}{2} \sigma^{2}\frac{\partial^{2} u}{\partial x^{2}} = 0,$$
with the terminal condition: $u(x,T)=\frac{Z(x,T)}{P(T,T_{n})(x)}$.

We apply the Crank-Nicolson Scheme to discretize the PDE:
\begin{align}
    &\frac{\partial u}{\partial t} = \frac{u_{i}(t_{j})-u_{i}(t_{j-1})}{\Delta t} \\
    &\frac{\partial u}{\partial x} =\frac{1}{2}\frac{u_{i+1}(t_{j})-u_{i-1}(t_{j})+u_{i+1}(t_{j-1})-u_{i-1}(t_{j-1})}{2\Delta x}\\
    &\frac{\partial^{2} u}{\partial x^{2}} =\frac{1}{2} \frac{\overbrace{[u_{i+1}(t_{j})+u_{i-1}(t_{j})-2u_{i}(t_{j})]}^{y(t_{j})}+[y(t_{j-1})]}{(\Delta x)^{2}}
\end{align}
Hence, the solution vector U satisfies: 
$$AU^{i}=MU^{i+1}+b.$$

The pricing of Bermudan options $(t \in \{0=t_{0},t_{1},...,t_{N}=T\})$ directly stems from that of European options:
\begin{itemize}
    \item The first step involves solving the PDE over the interval $[t_{N-1},t_{N}]$.
    \item At $t_{N-1}$, we have 
    $$ \pi(x,t_{N-1}) = \max\{Z_{t_{N-1}}(x),\mathbb{E}^{\mathbb{Q}^{T_{n}}}[\frac{Z_{t_{N}}}{P_{t_{N},T_{n}}}|X_{N-1}=x]*P_{t_{N-1},T_{n}}\}$$
    where the conditional expectation is determined by the solution vector.
    \item Subsequently, we iteratively solve the equation over distinct intervals $[t_{i}, t_{i+1}]$ comparing the immediate exercise value with the continuation value at each exercise date.
\end{itemize}
\subsubsection{Rogers' method}
The concept of Rogers' method is rooted in the dual formulation of option pricing. However, the selection of the martingale differs from the approach we have introduced. The procedure unfolds as follows:
\begin{itemize}
    \item We choose this specific martingale $M$ is:$$dM_{t} = d\tilde{P}(t,X_{t}),$$ where $\tilde{P}(t,X_{t})$ is the discounted price of the corresponding European option at time $t$.
    \item Subsequently, we endeavor to identify the relatively optimal linear form of this martingale in relation to the dual formula. $$\inf_{\lambda} \mathbb{E}[\sup_{t\leq T}(\beta_{t}^{-1}Z(X_{t})-\lambda M_{t})].$$
    \item Consequently, our objective shifts towards finding the minimizer $\lambda$.
\end{itemize}

\subsection{Results}
In this section, we will unveil the prices obtained through various methodologies for the Bermudan Floor under the Hull-White model. In addition, we subject these methods to testing both on 5 and 50 exercise dates.

In the subsequent example, the parameters are as follows: Maturity $T=10$, Strike $K = 0.03$, $r=0.005$, $\sigma = 0.005$, $\lambda = 0.02$, $\theta = 1$, $T_{n} = T+\theta$. 
During the implementation, we simulate $10^{5}$ paths. For training the neural networks, we follow the same methodology as in the previous example for hyperparameter tuning. Specifically, the neural network architecture is structured as [(50, 50), (50, 30, 30)]. The learning rate is set at $0.0004$, the batch size is chosen as $800$, and the training is conducted over $150$ epochs.

As demonstrated in Table 3.2, it is evident that with an increase in the number of exercise dates, there is a corresponding augmentation in both the lower and upper bounds of the option price. Consequently, the differences between the upper and lower bounds of the price diminishes.
\begin{table}[!h]
\centering
    \begin{tabular}{lccc}
        \hline
        &Lower Bound &Upper Bound  &Difference\\
        \hline
        5 exercise dates &0.024758  &0.024855 &9.73785e-05\\
        \hline
        50 exercise dates &0.024872  &0.024912  &3.995538e-05\\
        \hline
    \end{tabular}
    \caption{Results on Bermudan Floor}
\end{table}

Tables 3.3, 3.4 and Figure 3.4 offer a comparative analysis between the established benchmarks and the algorithm outlined above. A notable observation is that the prices obtained through neural networks adeptly function as robust lower and upper bounds, outperforming Rogers' method. Intriguingly, the precision of prices achieved through neural networks increases with a higher count of exercise dates.

\begin{table}[!h]
\centering
\begin{tabular}{lcccc}
    \hline
    &L.B. &Approx. Value &U.B. &Diff. \\
    \hline
     NN &0.024758  &- &0.024855 &-\\
    \hline
     Rogers &-  &-  &0.024938 &+8.3144e-05(w.r.t U)\\
    \hline
     PDE &-  &0.024783  &- &+2.5212e-05(w.r.t L)\\
     \hline
\end{tabular}
\caption{Compare with Benchmarks - 5 exercise dates}
\end{table}

\begin{table}[!h]
    \centering
    \begin{tabular}{lcccc}
        \hline
        &L.B. &Approx. Value &U.B. &Diff. \\
        \hline
        NN &0.024872  &- &0.024912 &-\\
        \hline
        Rogers &-  &-  &0.024999 &+8.648e-05(w.r.t U)\\
        \hline
        PDE &-  &0.024794  &- &+7.249e-06(w.r.t L)\\
        \hline
    \end{tabular}
    \caption{Compare with Benchmarks - 50 exercise dates}
\end{table}

\begin{figure}[!h]
    \centering  
    \subfigure{
    \label{ci1}
    \includegraphics[width=0.46\textwidth]{images/CI1.png}}
    \subfigure{
    \label{ci2}
    \includegraphics[width=0.46\textwidth]{images/CI2.png}}
    \caption{Visualization Confidence Intervals}
\end{figure}
Continuing within the context of 5 exercise dates, we now vary the strike values and plot the resulting prices.
\begin{figure}[!h]
    \includegraphics[scale=0.55]{images/floor_5_vary_K.png}
    \caption{Prices vary with K}
    \label{K}
\end{figure}

Indeed, as shown in the Figure \ref{K}, the neural network algorithm exhibits a strong and robust performance similar to that of traditional methods, except for the case where K is particularly small.

\section{Generalized Models}
Let's continue discussing the context of Bermudan Floor within the Hull-White model. In the previous examples, the spot rate $r$ and volatility $\sigma$ were treated as fixed constants, which may not be entirely practical in real-world applications. To address this limitation, we introduce a more general model. In the data simulation phase, we introduce randomness to both $r$ and $\sigma$, considering them as integral components of the input data. The conceptual representation of this idea is illustrated in the Figure \ref{nn2}.
\begin{figure}[!h]
    \centering
    \includegraphics[scale=0.45]{images/NN2.png}
    \caption{Neural Networks with $\sigma$ and r as inputs}
    \label{nn2}
\end{figure}

Furthermore, we proceed to plot graphs where we keep $r$ fixed while varying $\sigma$, as well as keeping $\sigma$ fixed while varying $r$. 
All other parameters remain unchanged, maturity $T=10$, Strike $K = 0.03$, $\lambda = 0.02$, $\theta = 1$, $T_{n} = T+\theta$.

As evident from Figures \ref{sigma} and \ref{r}, despite some loss of precision compared to the initial models, however, our generalized model is still robust and consistently generates reliable prices.
\begin{figure}[!h]
    %\centering
    \includegraphics[scale=0.52]{images/sigma.png}
    \caption{Prices vary with $\sigma$ (r = 0.005)}
    \label{sigma}
\end{figure}

\begin{figure}[!h]
    \centering
    \includegraphics[scale=0.52]{images/r.png}
    \caption{Prices vary with r ($\sigma$ = 0.04)}
    \label{r}
\end{figure}
\end{comment}
