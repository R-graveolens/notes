\chapter{Calibration problem in MOT formulation}
\section{Arbitrage constraints}
Arbitrage refers to a costless trading strategy that offers a positive probability of 
earning a risk-free profit. There are two main types: static and dynamic arbitrage. 
Static arbitrage involves exploiting opportunities through fixed positions in options and the 
underlying stock at the initial time, with the option to adjust the underlying position at a 
finite number of trading times in the future. In contrast, dynamic arbitrage relies on the 
dynamic and path-dependent properties of tradable assets, allowing continuous adjustments over 
time. Given that our method is model-independent, we focus solely on static arbitrage, where no 
modeling of asset dynamics is required.

Initially, we do not incorporate the price of the CMS forward. From the perspective of linear 
programming, by treating the swaption prices at different strikes as discrete data points, 
we can establish the arbitrage-free shape constraints for the call price surface. This ensures 
that the call price surface remains consistent with the principles of no-arbitrage across the 
range of strike prices. Since there is no concern for calendar arbitrage in our context, 
we focus on the following conditions to ensure arbitrage-freeness in our case.

A model \(\mathbb{M}\) is characterized as a filtered probability space 
$(\Omega,\mathcal{F},\{\mathcal{F}_{t}\}_{t\in\mathcal{T}},\mathbb{P})$
that includes an adapted diffusion process \(\{(R_{t},\mathbf{C}_{t})\}_{t\in\mathcal{T}}\). 
In this framework, \(\mathbf{C}_{t}\) denotes the prices of \(N = |\mathcal{K}|\) options at time \(t\), 
and we initially observe \(\mathbf{C}_{0}\). 
The set \(\mathcal{T}\) signifies the discrete trading times for the underlying asset, 
which in this case is \(\mathcal{T} = \{0, T_{f}\}\).
The First Fundamental Theorem of Asset Pricing (FFTAP) establishes a fundamental link between the 
no-arbitrage principle (both static and dynamic) and the existence of an equivalent martingale measure (EMM). 
Since the seminal work of Harrison and Kreps [22], different formulations of the FFTAP and broader concepts 
of no-arbitrage have emerged. In general terms, for a given model \(\mathbb{M}\), there is no arbitrage 
if and only if $\exists \mathbb{Q} \sim \mathbb{P}$, such that
\[
\forall(T,K)\in (\mathcal{T} \times \mathcal{K}),\ D(t)C_{t}(T, K)=D(s)\mathbb{E}^{\mathbb{Q}}[C_{s}(T,K)|\mathcal
{F}_{t}]
\]
for all \(t < s \leq T\), where \(t, s \in \mathcal{T}\). 
Consequently, static arbitrage constraints result from the relationships between terminal payoffs, 
discounted to the present moment.

Recall that, under the swap measure, the price of a payer swaption is given by: ..., and in this context, 
the forward swap rate behaves as a martingale. Hence, our goal—constructing an arbitrage-free implied volatility 
model—is equivalent to finding a martingale probability measure 
\(\mu^{*} \in \mathcal{P}(\{R_{0}\} \times \mathbb{R})\) that precisely matches these market prices. 
These probabilistic constraints can be expressed as: $...$. 

Furthermore, the equivalence between the shape constraints on prices and the constraints on the 
probabilistic measure has been established.

Another criterion for checking arbitrage involves plotting the RND. 
According to Breeden and Litzenberger, the second partial derivative of the call price with respect to the 
strike price represents the RND of \( R_1 \), up to a discount factor. 
Specifically, it can be expressed as:
$$
\text{pdf}(R_1) = \frac{1}{LVL(0)}\frac{\partial^{2} C_{t}}{\partial K^{2}},
$$
To ensure the absence of arbitrage, the RND must be a true probability density function, 
meaning it should not take on negative values. 
We will further present this as existence of arbitrages in the experiments chapter.

\section{MOT problem formulation}

To build such an element belonging to \(\mathcal{M}\), we begin by selecting a prior probability measure \(m_0\) 
on \(\mathcal{P}(R_{0} \times \mathbb{R})\), which represents the space of all possible measures on the pair 
\((R_0, R_1)\). We then seek a measure \(\mu \in \mathcal{M}\) that minimizes the "distance" to \(m_0\), 
as illustrated in the accompanying graph. 

The "distance", in more rigorous mathematical expression, is the objective function $\mathcal{F}$, so our objective
is to solve the optimization problem $inf_{\mu \in \mathcal{M} \mathcal{F}_{m_0}}(\mu) = \mathcal{F}_{m_0}(\mu^{*})$.

\section{CMS forward joint calibration problem}
Now, we study the joint calibration problem by incorporating the price of the CMS forward into our considerations. 
From this point on, we assume that the floating interest rate in the swaptions we handle is based on the risk-free rate, EST. 
This allows us to use the simplified formula for the swap rate and related calculations. Additionally, we assume that the payment date \(T_p\) 
coincides with the expiry date \(T_f\).

The first challenge is that the price of the CMS forward is evaluated under the risk-neutral forward measure \(Q_{T}\), 
rather than the swap measure \(Q_{LVL}\). Therefore, we begin by changing the probability measure from the numeraire \(B(t, T_f)\) 
to the numeraire \(LVL_t\). According to the change of numeraire theory, \( \ldots \). 

However, the dependency of formula 1 with respect to \(R_1\) is not immediately clear, so by applying the conditional expectation tower 
property, we can express it as \( \ldots \). We will denote this as \(G(R_1) = \ldots \). The function \(G\) depends on the assumptions we 
make regarding the dynamics of the rate curve.

Here, we introduce the Ho-Lee model, where the dynamics of Zero-Coupon bonds (ZCs) are as follows:
\[ \ldots \]
In this model, \(r_t\) represents the instantaneous short-term rate, \(W\) is a Brownian motion under the risk-neutral measure, \(\lambda\) 
is the mean reversion parameter, and \(\sigma\) is the volatility function.

Because, in our view, the CMS forward price is significantly more important than swaption prices, we aim to ensure that the adjusted CMS 
forward price after calibration remains as close to its original value as possible. Additionally, in real markets, swaptions are typically 
over-the-counter (OTC) products, but due to their relatively high liquidity and active market participation, bid and ask quotes are available 
in certain markets, particularly for standardized and highly liquid contracts. In contrast, CMS forward contracts are less transparent, 
and their pricing information is less readily available. Therefore, we can develop two different approaches to incorporate the CMS forward 
into our calibration process.

\subsection{Inequality constraint} 
The first approach is to treat the CMS forward price as the mid-price in our input and manually add bid-ask spreads, 
typically by a few basis points. We then consider \(CMS^{\text{ask/mid/bid}}\) as an additional swaption price and incorporate 
it into the constraints of the probability set as well as into the objective function \(\mathcal{F}\).

The reformulated optimization problem thus becomes:

\[
\min_{\mu \in \mathcal{M}} \mathcal{F}_{m_0}(\mu) 
\]
where 
\begin{align*}
    \mathcal{M} = \{ \} \\
    \mathcal{F} = 
\end{align*}

Following the same derivations as outlined in above section, we arrive at the following final optimization problem:

As for the equation system to cancelling the gradient of this strictly convex function, there is an additional equation for $V_{CMS}$.

\subsection{Equality constraint}
The second approach is to take the CMS forward price as a equality constraint. Recall the significantion of the weights, the smaller is, the 
more important the correspond swaption is, and when the $\omega$ goes to 0, the contracts required between the bid-ask spreads becomes to 
a more strong equality constraint on mid prices. And here we directly apply the equality constraint on CMS forward price. 
In this way, the constrained probability set is written as $...$. During the similar deducation of the final optimization problem, 
notice that the new equality constraint will introduced a new lagrange multiplier $g_0$. And the problem is as follows:...

The second approach put more emphasis 

