\chapter{Introduction}
Developing arbitrage-free implied volatility surfaces from bid-ask price quotes is a key challenge in the financial industry, 
especially for pricing exotic options. These surfaces are essential in risk-neutral models, 
which rely on vanilla option prices for calibration. Without an accurate, arbitrage-free volatility surface, 
pricing models may suffer from inconsistencies, leading to mispricing and unreliable risk management.
\begin{comment}
Developing arbitrage-free implied volatility surfaces from bid-ask price quotes has been a longstanding challenge in the 
financial industry, particularly in the pricing of exotic options. These options rely on risk-neutral models, such as the 
local volatility and local stochastic volatility models, which are calibrated to vanilla option prices. Both models require 
accurate, arbitrage-free volatility surfaces to ensure precise pricing and effective risk management. These surfaces are 
crucial for avoiding inconsistencies that could lead to mispricing, preserving the integrity of risk-neutral probability 
measures, and maintaining stability in option pricing frameworks. Moreover, arbitrage-free surfaces provide reliable 
calculations of option sensitivities (Greeks), essential for constructing hedging strategies and managing financial risks. 
Additionally, they promote model stability by ensuring that simulations and calculations converge to meaningful results.
\end{comment}
To address these challenges, various methods have been proposed in the literature. In this study, we explore a non-parametric 
calibration method based on the theory of martingale optimal transport (MOT). This method has proven effective in producing 
arbitrage-free implied volatility surfaces, fitting market quotes within bid-ask spreads, 
and accurately capturing complex volatility smile patterns, such as Mexican hat-shaped curves. 

\vspace{.1in}
Despite the limitations of the Black-Scholes-Merton (BSM) model in reflecting real market conditions, 
it remains widely used due to its practicality in transforming option prices into implied volatility, 
a single value that provides valuable market insights. Implied volatility helps practitioners assess market sentiment and risk.

When implied volatilities are plotted against strike prices for a given maturity, 
common patterns like skew or smile often appear. 
These patterns reflect deviations from normality in the underlying return distribution: 
a smile indicates fat tails, while a skew signals asymmetry.

Variations in the implied volatility smile across different strikes and maturities provide insights into how the return 
distribution evolves over time. This makes the collection of vanilla option prices across different strikes and maturities 
critical for financial applications such as model calibration, pricing, and risk-neutral density (RND) estimation. 
However, if these price data contains arbitrage, it can result in inaccurate model calibration and erroneous RND estimates. 
Most derivative pricing models are designed to be arbitrage-free to avoid economically unrealistic outcomes. 
For example, models like Dupire's local volatility model fail to calibrate correctly when the data contains arbitrage, 
and arbitrage-free interpolation methods also break down under these conditions.

To improve the accuracy and robustness of pricing models, it is essential to remove arbitrage from the data. 
Two common techniques for achieving this are smoothing and filtering, and in this study, we focus on smoothing methods. 
Significant contributions to smoothing have been made by Ait-Sahalia and Duarte, Fengler, Gatheral, and Jacquier. 
In recent years, stochastic volatility models (SVMs) have gained a lot attention in this context. 
In practice, smoothing typically involves calibrating models to create a smooth \(C^{1,2}\) call price function 
\((T,K)\mapsto C(T,K)\), or equivalently, an implied volatility function \((T,K)\mapsto\sigma_{\text{imp}}(T,K)\), 
which results in arbitrage-free surfaces.

While existing calibration methods offer advantages, they also have some limitations. 
For instance, SVM parameter calibration can be time-consuming, especially without a closed-form solution, 
and the non-convex optimization involved may not always converge. Additionally, 
although parametric models like SVI are easy to implement, they do not always ensure arbitrage-free surfaces across both strike prices and maturities.

\vspace{.1in}
Instead of using parametric calibration models, this study addresses the arbitrage-free calibration problem with a 
non-parametric discrete-time model based on martingale optimal transport (MOT) theory. Optimal transport theory has 
recently attracted significant attention. Initially introduced by Monge [20] in civil engineering, it was later formalized 
with a mathematical foundation by Kantorovich [19], who introduced the dual problem framework. In recent years, 
optimal transport theory has been applied to robust hedging and pricing in both discrete and continuous-time models 
[2, 17, 21, 7]. It was discovered that pricing bounds for path-dependent derivatives using fixed European options could 
be framed as a martingale optimal transport problem. This led to MOT becoming a field of study in its own right, 
with the theory further applied to the non-parametric discrete-time model proposed by Guyon [14], 
particularly to address the VIX/SPX calibration problem.

This approach is equivalent to identifying an element from the non-empty set of probability measures \(\mathcal{M}\). 
To construct this model, we begin by selecting a prior probability \(\bar{\mu} \in \mathcal{P}\) based on market information 
and then search for the model \(\mu\) within \(\mathcal{M}\) that is closest to \(\bar{\mu}\) in terms of entropy, minimizing 
the relative entropy with respect to \(\bar{\mu}\), using the Kullback-Leibler divergence in the objective function. 
Several dual versions of the problem can be formulated, and Guyon demonstrated that one of these has a clear financial 
interpretation through exponential utility indifference pricing. He also proved there is no duality gap between the primal 
and dual problems. The minimum entropy model is explicitly characterized by the minimizers of the dual problem.

More specifically, the minimum entropy model is described non parametrically by its Radon-Nikodym derivative with respect to 
\(\bar{\mu}\), expressed as the exponential of a combination of payoffs from the underlying instruments. In the context of 
equities, this portfolio consists of the same hedging instruments used in the super-replication problem. To numerically solve 
the dual problem, we applied the Sinkhorn algorithm [81], which was revived by Cuturi [24] for classical optimal transport 
and later adapted by De March and Henry-Labordère [26] for fast construction of arbitrage-free smile interpolations. 
The algorithm iteratively optimizes in different directions until it converges at the global optimum, where the gradient of 
the dual objective function vanishes.

It is worth noting that the dual formulation and another numerical method were explored much earlier in [1] 
within the context of uncertain volatility calibration using entropy minimization. Additionally, calibration within 
bid-ask spreads was investigated through the introduction of an \(L^{2}\) weighted penalization [2]. 
However, at that time, the connection to optimal transport and the proof of the duality result were not considered.

\vspace{.1in}
Our study is largely inspired by the work of De March, Henry-Labordère, and Guyon. However, neither De March nor 
Henry-Labordère mentioned the property of eliminating arbitrage. Regardless of the input data, the posterior density, 
if it exists, is truly arbitrage-free due to the positivity ensured by the minimum entropy probability formula. In Guyon's 
study, he applied strict equity constraints on prices and showed that if there are arbitrages in the price data, 
the objective function value will approach infinity and the algorithm will not converge. However, similar to the work of 
Allenaveda, De March, and Henry-Labordère, if we consider bid-ask quotes and apply \(L^{2}\) penalization, 
a solution can still be found even if some arbitrage exists in the input price data.

Another difference is that we applied this approach in the realm of interest rate derivatives, rather than equities, 
where calendar arbitrage is not considered. Towards the end of our study, we explored joint calibration using both vanilla 
swaption prices and CMS forwards, where the latter product's price depends on the entire market distribution, making the 
calibration problem more complex.

The rest of this paper is structured as follows: After introducing our discrete-time setting in Section 2, 
we discuss how to formulate the arbitrage-free calibration problem in Section 3. In Section 4, we derive 
the Sinkhorn algorithm used to numerically solve these problems, along with some implementation tricks. 
Section 6 presents the results of several experiments, and Section 7 concludes.


